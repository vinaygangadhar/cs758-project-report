\begin{abstract} 

The slowing of Moore’s law and Dennard scaling is limiting the performance 
improvements of single core processors. 
Increasing clock frequency any farther will lead to 
high leakage current and infeasible power consumption. Further, 
these cores are extra-ordinarily inefficeint optimzied only for single thread
arbitrary programs. Over the past decade, focus has been shifted to multicore processors 
to increase throughput by having multiple cores  to target
different types of parallelism - instructoion (ILP), data (DLP) and  thread (TLP).
However, even the multi-core processors are not scalable for parallelization beyond a point 
due to Amdhal's law. To address the challenges of rising dark silicon and the end
of Dennard scaling, architects have turned to heteregenous architectures 
with special purpose domain specific accelerators (DSAs), for higher performance and energy effieincy.
While providing huge benefits, DSAs are prone to obsoletion due to domain volatility,
have recurring design and verification costs, and have large
area footprints when multiple DSAs are required in a single
device to reap out the benefits of different applcaition acceleration.
To attack such problems of DSAs, while still having  


Because of the benefits of generality, this work explores
how far a programmable architecture can be pushed,
and whether it can come close to the performance, energy,
and area efficiency of a DSA-based approach.
Our insight is that DSAs employ common specialization
principles for concurrency, computation, communication,
data-reuse and coordination, and that these same principles
can be exploited in a programmable architecture using
a composition of known microarchitectural mechanisms.
Specifically, we propose and study an architecture called
LSSD, which is composed of many low-power and tiny
cores, each having a configurable spatial architecture,
s



\end{abstract}


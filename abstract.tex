\begin{abstract} 

The slowing of Moore’s law and Dennard scaling is limiting the performance 
improvements of single core processors. 
Increasing clock frequency any farther will lead to 
high leakage current and infeasible power consumption.
Over the past decade, focus has been shifted to multicore processors 
to increase throughput by having multiple cores  to target
different types of parallelism - instructoion (ILP), data (DLP) and  thread (TLP).
However, even the multi-core processors are not scalable for parallelization beyond a point 
due to Amdahl's law. To address these challenges of rising dark silicon and the end
of Dennard scaling, in recent years architects have turned to heteregenous architectures 
with special purpose domain specific accelerators (DSAs), for higher performance and energy effieincy.
While providing huge benefits, DSAs are prone to obsoletion due to domain volatility,
have recurring design and verification costs, and have large
area footprints when multiple DSAs are required in a single
device to reap out the benefits of different applcaition acceleration.
To attack such problems of DSAs and multi-core processors, 
while still retaining programmbility of general prupose processors and 
efficiency of DSAs, there is an on-going research in Vertical Research Group, University of Wisconsin-Madison
aiming to build a general purpose multi-tile programmable accelerator called Proximate. 

This project, focuses on exploring the parallel programming design space
of Proximate, a multi-tile programmable hardware accelerator.
The aim is to investiagate the programming interface  of Proximate



the multi-tile architecturescalability and speedup of such accelerators
compared to a traditional server class multi-core processors for parallel programs.
Proximate is aimed at targeting
both regular and irregular class of workloads, and our project tries to do a 
quantittaive and qualitaitve comaprative study of scalbility of such parallel programs.
We compare the parallel implementation of different classes of wrokalods (pthread, OpenMP) run on 
a 64 core Xeon-Phi
to the Proximate programs written in its own programming model. 



\end{abstract}


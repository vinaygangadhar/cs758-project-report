\begin{abstract} 

The slowing of Moore’s law and Dennard scaling is limiting the performance 
improvements of single core processors. 
Increasing clock frequency any farther will lead to 
high leakage current and infeasible power consumption.
Over the past decade, focus has been shifted to multi-core processors 
to increase throughput by having multiple cores  to target
different types of parallelism - instruction (ILP), data (DLP) and  thread (TLP).
However, even the multi-core processors are not scalable for parallelization beyond a point 
due to Amdahl's law. To address these challenges of rising dark silicon and the end
of Dennard scaling, in recent years architects have turned to heterogeneous architectures 
with special purpose domain specific accelerators (DSAs), for higher performance and energy efficiency.
While providing huge benefits, DSAs are prone to obsoletion due to domain volatility,
have recurring design and verification costs, and have large
area footprints when multiple DSAs are required in a single
device to reap out the benefits of different application acceleration.
To attack such problems of DSAs and multi-core processors, 
while still retaining programmability of general purpose processors and 
efficiency of DSAs, there is an on-going research in Vertical Research Group, University of Wisconsin-Madison
aiming to build a general purpose multi-tile programmable accelerator called Proximate. 

This project focuses on exploring the parallel programming design space
of Proximate, a multi-tile programmable hardware accelerator.
The aim is to investigate the programming interface  of Proximate, the parallel
hardware improvements and in general explore the parallel programs run on proximate.
We have tried to explore a new type of the multi-tile programmable architecture,
its scalability and speedup of such accelerator architecture 
compared to a traditional server class multi-core processors for parallel programs.
In summary, we explored different programming design points in the project and proximate is able to achieve speedups of
50-100x over a traditional server class multi-core processor. We try to analyze this trend over the coarse of this
document by explaining the architecture, programming model and the performance results of proximate. 

\end{abstract}


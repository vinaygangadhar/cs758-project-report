\section{Conclusion and Future Work} \label{sec:conc}
This project focused on exploring the parallel programming design space
of Proximate, a multi-tile programmable hardware accelerator.
The aim is to investigate the programming interface  of Proximate, the parallel
hardware improvements and in general explore the parallel programs run on proximate.
We have tried to explore a new type of the multi-tile programmable architecture,
its scalability and speedup of such accelerator architecture 
compared to a traditional server class multi-core processors for parallel programs.
In summary, we explored different programming design points in the project and proximate is able to achieve speedups of
50-100x over a traditional server class multi-core processor. 

Proximate is a programmable accelerator that targets high 
efficiency and programmability. It is a compute engine 
for both regular and irregular workloads.

Proximate exposes a flexible  programming model with a task-based programming API. 
The programmer manages data sharding to achieve task and data locality for performance and scalability. 
The simple queueing model supports more concurrency with locality 
as each worker thread will only execute tasks in their own queue. 
Preliminary results shows that Proximate outperforms current 
multiprocessor programming paradigms by a factor of 10x for regular 
workloads and a factor of 2x for irregular workloads. Although
multi-threaded SoftBrain may cause new issues such as synchronization, 
schedule, and saturate memory bandwidth, it could uncover new bounds of performance. We leave multi-threaded SoftBrain to future work. 


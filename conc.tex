\section{Summary} \label{sec:summary}
This dissertation will address the challenges of rising dark silicon 
and the end of Dennard scaling through researching methods for understanding,
developing, and employing accelerators.
To enable this, completed work has developed the transformable 
dependence graph (TDG), which can model acceleration at a higher 
level of abstraction while still capturing important
interactions with the general purpose processor and the applications.
The research will use this abstraction in three areas:  First,
to decipher accelerators'  fundamental limitations and opportunities, 
which could help accelerator designers focus their efforts.  
Second, to enable accelerator design practices which are fast and explore
a broad design space.  And third, to enable the study of practical 
techniques for scheduling multi-accelerator systems.  

The three pieces of prosed work are far from independent.  Part of the effectiveness
of the TDG-based design strategy comes from the ability to perform limit studies.  By
enabling multi-accelerator scheduling on applications, we may open new limits
and opportunities for acceleration.  More concretely, the NLA design we are proposing
can be viewed as breaking the limitations of the BERET architecture, and the
architectures we develop using our design strategy will help to make
multi-acceleration useful.  Taking the components of the proposed work together,
the goal of the proposed research is to enable the development of
accelerator systems which can provide order-of-magnitude better performance and
energy-efficiency.
